\documentclass[]{scrreprt}
%\usepackage{amsmath,amsfonts,graphicx}
%\usepackage{multirow}
%\usepackage{pslatex}
%\usepackage{tabularx}
%\usepackage{comment}
%\usepackage{xspace}
%\usepackage{array}
%\usepackage{hyperref}
\usepackage{caption}

\RequirePackage{fix-cm}
\usepackage{graphicx}
\usepackage{color,soul}
\usepackage{todonotes}
\usepackage{subfigure}
\usepackage{amsmath}
\usepackage{natbib}


\DeclareCaptionFont{white}{\color{white}}
\DeclareCaptionFormat{listing}{\colorbox{gray}{\parbox{\textwidth}{#1#2#3}}}

\graphicspath{
{figures/}
}

\newcommand{\uo}{\mbox{UO\textsubscript{2}}\xspace}

\setcounter{secnumdepth}{3}

\newcommand{\moose}{{MOOSE}}
\newcommand{\redback}{{REDBACK}}
\newcommand{\pd}{\ensuremath{\partial}}
\newcommand{\pdiff}[2]{\ensuremath{\pd_{#2} #1}}

\begin{document}


\title{Redback Theory Manual}
\author{Thomas Poulet \\ CSIRO 
	\and Manolis Veveakis \\ UNSW}
\maketitle

\tableofcontents

%%%
\chapter{Introduction}

The \redback{} application was developed to model multi-physics \textit{Rock mEchanics with
Dissipative feedBACKs} in a tightly coupled manner. It is based on the Multi-physics Object Oriented Simulation Environment \moose{}\footnote{http://mooseframework.org} \citep{Gaston2009} which proposes a powerful and flexible platform to solve multi-physics problems implicitly and in a tightly coupled manner on unstructured meshes. \moose{} aims at providing a wide range of modules to model various physical phenomena, including rock mechanics, which are as flexible as possible and can be easily coupled together. By comparison, \redback{} is  
The philosophy behind \redback{} is to focus on a non-dimensional formulation of the problem in order to focus on the physical processes at play

Please cite...


\chapter{Governing equations}
\label{chapter:gov_eqs}



The system in its final form is
\begin{subequations}
\label{eq:final_system_of_equations_dimensionless}
\begin{align}
  0 &= \pdiff{\sigma^{\prime}_{ij}}{j} + \pdiff{\Delta p_f}{i} + b_i, \\   
  0 &= \pdiff{\Delta p_f}{t} + Pe\: v^p_i \pdiff{\Delta p_f}{i} -
  Pe\:v^T_i \pdiff{T}{i} - \pdiff{\left[\frac{1}{Le} \pdiff{\Delta
  p_f}{i} \right]}{i} \\ \nonumber
  ~ & \qquad\qquad - \Lambda \pdiff{T}{t} +
  \frac{\dot{\epsilon_V}}{\bar{\beta}\:\sigma_{ref}} -\frac{1}{Le_{chem}}
  \omega_F, \\
  0 &= \pdiff{T}{t} + Pe\:\bar{v_i}\pdiff{T}{i} - \pd_{ii}^2 T - Gr
  \: \sigma_{ij}^{pl}\dot{\epsilon}_{ij}^{\,pl}  + Da_{endo}\: \omega_F -
  Da_{exo}\: \omega_R.
\end{align}
\end{subequations}
All dimensionless groups are defined in Tab.~\ref{tab:dimensionless_nbs} and
\begin{align*}
  \bar{\beta} &= (1-\phi)\beta_s + \phi\beta_f, \\
  v_i^p &= (1-\phi)\frac{\beta_s}{\bar{\beta}} v_i^s + 
    \phi\frac{\beta_f}{\bar{\beta}}v_i^f, \\
  v^T_i &= (1-\phi)\frac{\bar{\lambda}_s}{\bar{\beta}} v_i^s + 
    \phi\frac{\bar{\lambda}_f}{\bar{\beta}} v_i^f,\\
  \bar{v_i} &= (1-\phi) v_i^s + \phi v_i^f, \\
  \omega_F &= (1-\phi)(1-s)\exp\left(\frac{Ar_F \: \delta T}{1 + \delta T}
  \right), \\
  \omega_R &= (1-\phi)\:s\:\Delta \phi_{chem}\exp\left(\frac{Ar_R \: \delta
  T}{1 + \delta T} \right), \\
  \dot{\vec{\epsilon}}^{pl} &= \dot \epsilon_0 \: \exp\left( \frac{Ar \: \delta
  T}{1 + \delta T}\right) \sqrt{ \left\langle\frac{q -
  q_Y}{\sigma_{ref}}\right\rangle^{2m} + \left\langle\frac{p
  - p_Y}{\sigma_{ref}}\right\rangle^{2m}}.
\end{align*}



The total porosity $\phi$ is expressed as the sum of its initial value, $\phi_0$,
and the newly created interconnected pore volume. Pore volume can be created
by mechanical ($\Delta\phi_{mech}$) and chemical ($\Delta \phi_{chem}$)
processes such that the total porosity reads
\begin{eqnarray}
    \label{eq:porosity}
    \phi = \phi_0 + \Delta\phi_{mech} + \Delta\phi_{chem} = \frac{V_{B}}{V},
\end{eqnarray}
where $V_B$ is the volume occupied by fluid $B$.  The evolution of mechanical
porosity contains two components, a plastic part $\Delta
\phi^{pl}_{mech}=\epsilon^{pl}_V$, with $\epsilon^{pl}_V$ the volumetric plastic strain, and an
elastic one $\Delta \phi^{e}_{mech}=(1-\phi)\left( \beta_s \Delta p_f -
\lambda_s \Delta T \right)$
where $\beta_s$ and $\lambda_s$ are compressibility and
thermal expansion coefficients of the solid phase, respectively.



\section{Chemical damage}
\label{subsec:chem_damage}

Thermally activated chemical reactions are allowed to take place and
in this work we concentrate on (de-)hydration reactions of the form
\begin{equation}
  \label{eq:reaction}
  \nu_1 AB_{s} 
  \overset{\omega_F}{\underset{\omega_R}{\rightleftharpoons}} \nu_2 A_{s} + \nu_3 B_{f},
\end{equation} 
where the subscripts $s$ and $f$ refer to solid and fluid phases and $\nu_i$
$(i=1,2,3)$ are stoichiometric coefficients.  The reaction
equation~\eqref{eq:reaction} states that the solid $A$ can release/bind the
component $B$ into/from the fluid phase which increases/reduces the pore pressure.

The kinetics of the decomposition reaction~\eqref{eq:reaction} are assumed
to follow a standard Arrhenius dependency on temperature 
\citep{Poulet2014}. As a result, the rates of the forward, $\omega_F$, and
reverse reaction, $\omega_R$ (let $\nu_1=\nu_2=\nu_3=1$) can
be expressed as \citep{Alevizos2014}
\begin{subequations}
\label{eq:reaction_rates_txt}
\begin{align}
  \omega_F &=  \frac{\rho_{AB}}{M_{AB}} (1 - \phi)(1 - s)  k_F e^{-Q_F/RT}, \\
  \omega_R &=  \frac{\rho_{A} \rho_{B}}{M_A M_B} (1 - \phi) s \Delta\phi_{chem}
  k_R e^{-Q_R/RT},
\end{align}
\end{subequations}
where $\rho_i$ and $M_i$ $(i = A, B, AB)$ are the densities and molar masses of
the respective constituent,
$k_F,k_R$, $Q_F,Q_R$ are the pre-exponential factors and activation enthalpies
of the forward and reverse reaction,
$\phi$ is porosity and $\Delta\phi_{chem}$ denotes change in porosity due to 
chemical processes. We define the solid ratio
\begin{eqnarray}
    \label{eq:s}
    s = \frac{V_{A}}{V_s} = \frac{V_{A}}{(1-\phi) V},
\end{eqnarray}
where $V$ is a representative volume, $V_A$ and $V_s$ is the volume of solid
phase $A$ and all solid within $V$, respectively. The solid ratio is a measure
of the extend of reaction~\eqref{eq:reaction}. Subsequently, the total
reaction rate is
\begin{equation}
 \label{eq:reaction_rate_total}
\omega = \left[(1 - s) - s  \Delta \phi_{chem} \frac{\rho_{A}
\rho_{B}}{\rho^2_{AB}} \frac{M^2_{AB}}{M_A M_B} K^{-1}_c e^{\Delta h/RT}
\right] (1 - \phi) \frac{\rho_{AB}}{M_{AB}} k_F e^{-Q_F/RT}
\end{equation}
where $K_c = k_F/k_R$ and $\Delta h = Q_R - Q_F$.  The expressions for the
dependency of the porosity $\phi$ and solid ratio $s$ on the reaction kinetics
are described in detail in \cite{Alevizos2014} and briefly summarized here.

We assume the following relations for the partial molar reaction rates of the
species involved
\begin{subequations}
\label{eq:js}
\begin{align}
  \omega_{AB} &= - \left[ \frac{\rho_{AB}}{M_{AB}} (1 - \phi)(1 - s)
  \right]^{\nu_1} k_F \exp(-Q_F/RT), \\
  \omega_A &= \left[ \frac{\rho_{A}}{M_A} (1 - \phi) s \right]^{\nu_2} k_A \exp(-Q_R/RT), \\
  \omega_B &= \left[ \Delta\phi_{chem} \frac{\rho_{B}}{M_B} \right]^{\nu_3} k_B
  \exp(-Q_R/RT),
\end{align}
\end{subequations}
and those rates are linked by the stoichiometry of the considered reaction~\eqref{eq:reaction} as
\begin{equation}
  \label{eq:j_equal}
	-\frac{\omega_{AB}}{\nu_1} = \frac{\omega_A}{\nu_2} = \frac{\omega_B}{\nu_3}.
\end{equation}
From Eqs.~(\ref{eq:js}-\ref{eq:j_equal}) and for $\nu_1=\nu_2=\nu_3=1$ we
derive the poro-chemical model
\begin{subequations}
\label{eq:s_phi_chem}
\begin{align}
  \Delta \phi_{chem} &= A_{\phi} \frac{1-\phi_0}{1 + \frac{\rho_{B}}{\rho_{A}} \frac{M_{A}}{M_{B}} \frac{1}{s}}, \\
  s &= \frac {\omega_{rel}} {1+\omega_{rel}},\\
  \omega_{rel} &= \frac{\rho_{AB}}{\rho_{A}} \frac{M_{A}}{M_{AB}} K_c \exp
  \left( \frac{\Delta h}{RT} \right),
\end{align}
\end{subequations}
where
$A_{\phi}$ is a coefficient that determines the amount of the
interconnected pore-volume (porosity) created due to the reaction. We assume
that all the fluid generated contributes to the interconnected pore volume, and
thus set $A_{\phi} = 1$.




\begin{table}
  \caption{Dimensionless parameters used in \redback{}. The coefficient $\delta$ is defined
such that $T^{\star} = (T-T_{ref})/(\delta T_{ref})$}
\label{tab:dimensionless_nbs}
\begin{tabular}{@{} l p{0.18\textwidth} l p{0.52\textwidth} @{}}
\hline\noalign{\smallskip}
Group & Name & Definition & Interpretation \\
\noalign{\smallskip}\hline\noalign{\smallskip}
$Gr$ & Gruntfest number & $\frac{\chi\sigma_{ref}\dot{\epsilon}_{ref} L^2_{ref}}{\alpha \delta T_{ref}}$ & ratio of mechanical rate converted into heat over rate of diffusive processes \\
$Da_{endo}$ & Endothermic Damk\"{o}hler number & $\frac{A_{endo} h_{endo} \rho_{AB} L^2_{ref}}{\alpha \delta T_{ref}}$ & ratio of endothermic reaction rate over rate of diffusive processes \\
$Da_{exo}$ & Exothermic Damk\"{o}hler number & $\frac{A_{exo} h_{exo} \rho_{AB} L^2_{ref}}{\alpha \delta T_{ref}}$ & ratio of exothermic reaction rate over rate of diffusive processes \\
$Ar$ & Arrhenius number & $Q_{mech}/(R T_{ref})$ & Ratio of activation energy over thermal energy \\
$Ar_F$ & Forward Arrhenius number & $Q_F/(R T_{ref})$ & Ratio of activation energy of forward reaction over thermal energy \\
$Ar_R$ & Reverse Arrhenius number & $Q_R/(R T_{ref})$ & Ratio of activation energy of reverse activation energy over thermal energy \\
$Le$ & Lewis number & $c_{th}/c_{hy}$ & Ratio of thermal over mass diffusivities \\
$Le_{chem}$ & Chemical Lewis number & $\frac{c_{th}\sigma_{ref}\beta_m }{L^2_{ref} A_{endo}}\left(\frac{\rho_{B}}{\rho_m} \right)\left(\frac{M_{AB}}{M_{B}} \right)$ & Ratio of thermal over chemical diffusivity of forward reaction \\
$\bar{\Lambda}$ & Thermal pressurisation coefficient & $\frac{\lambda_m}{\beta_m}\frac{\delta \:T_{ref}}{\sigma_{ref}}$ & Normalised thermal pressurisation coefficient, with $\lambda_m$ and $\beta_m$ the mixture thermal expansion and compressibility \\
$Pe$ & P\'{e}clet number & $L_{ref}\textbf{v}_m/c_{th}$ & Ratio of temperature advection rate over diffusion rate \\
$Pe_p$ & Pressure P\'{e}clet number & $L_{ref}\textbf{v}_p/c_{th}$ & Peclet number for the pressure effect in the pressure equation \\
$Pe_T$ & Temperature P\'{e}clet number & $L_{ref}\textbf{v}_T/c_{th}$ & Peclet number for the temperature effect in the pressure equation \\
\noalign{\smallskip}\hline
\end{tabular}
\end{table}

\bibliographystyle{aps-nameyear}      % American Physical Society (APS) style, author-year citations
\bibliography{redback}                % name your BibTeX data base
\nocite{*}


\chapter{Code architecture}
\section{Kernels}
\section{Porosity}
Porosity plays a particular role as its evolution depends on the mechanical, thermal, and hydraulic process models. As such, the total porosity evolution can not be handled within a material class unfortunately since it has components updated in more than one material. With the RedbackMechMaterial class already derived from the RedbackMaterial class, we can not create a dependency the other way around as it would create a circular dependency problem. There are at least two ways of working around that problem:
\begin{enumerate}
\item Porosity can be treated as an extra variable to solve for. 
\item Porosity can be treated as an AuxKernel, which allows us to update it with various components calculated in separate material, and yet have all materials use the total porosity (updated with delay obviously).
\end{enumerate}
In the first case the porosity evolution (and dependency on all process models) will be solved rigorously, but this will come at a greater computational cost. This is probably the neatest solution to handle the most generic case when porosity might be strongly dependent on all process models, but this is not the principal scenario we are aiming with the current development of Redback. We are indeed focusing on the case described in Sec.~\ref{chapter:gov_eqs}, where the porosity is much more strongly dependent on chemistry (which produces fluid and can therefore raise $\phi$ to $1$) than it is on temperature and pore pressure (inducing minor variations of $\phi$). As a result, we decided to implement the second option and handle the total porosity as an AuxVariable updated by an AuxKernel. This option provides more flexibility to compute the total porosity more or less accurately by updating the AuxKernel more or less frequently. At the moment we're treating the porosity in an explicit manner and only update it at the end of each step. This is equivalent to say that we neglect the mechanical update of the porosity during a single step and only consider its chemical variation (since it is the main evolution for the cases we consider).


\appendix

\chapter{Derivations}
This chapter documents some of the derivations used to obtain the equations presented in Sec.\ref{chapter:gov_eqs}.

\section{Mass balance}
We define the following densities
\begin{subequations}
  \label{eq:rho_1_rho_2}
  \begin{align}
    \rho_1 &= (1-\phi)\rho_s \\
    \rho_2 &= \phi \: \rho_f  
  \end{align}
\end{subequations}

and we use the usual material derivative definition
\begin{equation}
  \label{eq:material_derivative}
  \frac{D^{(i)}.}{D t} = \frac{\partial.}{\partial t} + v^{(i)}_k \frac{\partial.}{\partial x_k}
\end{equation}


\underline{Mass balance for the fluid phase}
\begin{equation}
  \label{eq:fluid_mass_balance_1}
  \frac{\partial \rho_2}{\partial t} + \frac{\partial( \rho_2 V^{(2)}_k)}{\partial x_k}= j_1
\end{equation}

Using Eq.~\ref{eq:rho_1_rho_2}b in Eq.~\ref{eq:fluid_mass_balance_1} we get
\begin{equation}
  \label{eq:fluid_mass_balance_2}
  \phi \frac{\partial \rho_f }{\partial t} + \rho_f\frac{\partial \phi }{\partial t} + \rho_f\frac{\partial( \phi V^{(2)}_k)}{\partial x_k}+ \phi V^{(2)}_k\frac{\partial\rho_f}{\partial x_k} = j_1
\end{equation}
Dividing by $\rho_f$ we obtain
\begin{equation}
  \label{eq:fluid_mass_balance}
  \frac{\phi}{\rho_f} \frac{D^{(2)} \rho_f }{D t} + \frac{\partial \phi }{\partial t} + \frac{\partial( \phi V^{(2)}_k)}{\partial x_k} = \frac{j_1}{\rho_f}
\end{equation}


\underline{Mass balance for the solid phase}
\begin{equation}
  \label{eq:solid_mass_balance_1}
  \frac{\partial \rho_1}{\partial t} + \frac{\partial( \rho_1 V^{(1)}_k)}{\partial x_k}= -j_1
\end{equation}
Using Eq.~\ref{eq:rho_1_rho_2}a in Eq.~\ref{eq:solid_mass_balance_1} we get
\begin{equation}
  \label{eq:solid_mass_balance_2}
  (1-\phi) \frac{\partial \rho_s }{\partial t} - \rho_s\frac{\partial \phi }{\partial t} + \rho_s\frac{\partial( (1-\phi) V^{(1)}_k)}{\partial x_k}+ (1-\phi) V^{(1)}_k\frac{\partial\rho_s}{\partial x_k} = -j_1
\end{equation}
Dividing by $\rho_s$ we obtain
\begin{equation}
  \label{eq:solid_mass_balance}
  \frac{(1-\phi)}{\rho_s} \frac{D^{(1)} \rho_s }{D t} - \frac{\partial \phi }{\partial t} + \frac{\partial(V^{(1)}_k)}{\partial x_k} - \frac{\partial( \phi V^{(1)}_k)}{\partial x_k} = -\frac{j_1}{\rho_s}
\end{equation}

\underline{Mass balance for the mixture (solid + fluid)}

Adding Eq.~\ref{eq:fluid_mass_balance} and Eq.~\ref{eq:solid_mass_balance} gives the mixture mass balance:
\begin{equation}
  \label{eq:mixture_mass_balance}
  \frac{(1-\phi)}{\rho_s} \frac{D^{(1)} \rho_s }{D t} +\frac{\phi}{\rho_f} \frac{D^{(2)} \rho_f }{D t} + \frac{\partial( \phi (V^{(2)}_k -V^{(1)}_k))}{\partial x_k}+ \frac{\partial(V^{(1)}_k)}{\partial x_k}  = \left(\frac{1}{\rho_f} - \frac{1}{\rho_s}\right)j_1
\end{equation}

\underline{Equation of state (EOS)}

\begin{equation}
  \label{eq:density_derivative}
  \frac{d\rho_{(i)}}{\rho_{(i)}} = \left( \frac{d\rho_{(i)}}{dp_f} \right)_T \frac{dp_f}{\rho_{(i)}} +\left( \frac{d\rho_{(i)}}{dT} \right)_p \frac{dT}{\rho_{(i)}}, \:\:\:i\in \{s,f\} 
\end{equation}

Using the definition for compressibility $\beta_{(i)}=\frac{1}{\rho_{(i)}}\left( \frac{d\rho_{(i)}}{dp_f} \right)_T$ and thermal expansion $\lambda_{(i)}=-\frac{1}{\rho_{(i)}}\left( \frac{d\rho_{(i)}}{dT} \right)_{p_f}$ we get the Equation of State (EOS)

\begin{equation}
  \label{eq:eos_def}
  \frac{d\rho_{(i)}}{\rho_{(i)}} = \beta_{(i)} dp_f - \lambda_{(i)} dT, \:\:\:i\in \{s,f\} 
\end{equation}

Using Eq.~\ref{eq:eos_def} in Eq.~\ref{eq:mixture_mass_balance} leads to
\begin{multline}
  \label{eq:mixture_mass_balance2}
  (1-\phi) \left[ \beta_s \frac{D^{(1)}p_f}{Dt} - \lambda_s\frac{D^{(1)}T}{Dt}  \right] + \phi \left[ \beta_f \frac{D^{(2)}p_f}{Dt} - \lambda_f\frac{D^{(2)}T}{Dt}  \right] \\
  + \frac{\partial( \phi (V^{(2)}_k -V^{(1)}_k))}{\partial x_k}+ \frac{\partial(V^{(1)}_k)}{\partial x_k}  = \left(\frac{1}{\rho_f} - \frac{1}{\rho_s}\right)j_1
\end{multline}
Rearranging the terms we get
\begin{multline}
  \label{eq:mixture_mass_balance3}
  \overbrace{\left[(1-\phi)\beta_s + \phi\beta_f\right]}^{\beta_m}  \frac{\partial p_f}{\partial t} 
  - \overbrace{\left[(1-\phi)\lambda_s + \phi\lambda_f\right]}^{\lambda_m} \frac{\partial T}{\partial t} \\
  + \left[(1-\phi)\beta_s V^{(1)}_k + \phi\beta_f V^{(2)}_k \right] \frac{\partial p_f}{\partial x_k} 
  - \left[(1-\phi)\lambda_s V^{(1)}_k + \phi\lambda_f V^{(2)}_k \right] \frac{\partial T}{\partial x_k} \\
  + \frac{\partial( \phi (V^{(2)}_k -V^{(1)}_k))}{\partial x_k}+ \frac{\partial(V^{(1)}_k)}{\partial x_k}  = \left(\frac{1}{\rho_f} - \frac{1}{\rho_s}\right)j_1
\end{multline}

\underline{Normalisation}

In order to deal with dimensionless parameters we introduce the following normalised variables
\begin{subequations}
  \label{eq:def_normalisations}
  \begin{align}
  p^* &= \frac{p_f}{p_{ref}}, \\   
  T^* &= \frac{T-T_{ref}}{\delta T_{ref}}, \\   
  x^* &= \frac{x}{x_{ref}}, \\   
  t^* &= \frac{c_{th}}{x^2_{ref}}t, \\   
  V^* &= \frac{V}{V_{ref}}.
  \end{align}
\end{subequations}

Dividing Eq.~\ref{eq:mixture_mass_balance3} by $\beta_m$ and switching to the normalised variables we get

\begin{multline}
  \label{eq:mixture_mass_balance4}
  \frac{p_{ref} \: c_{th}}{x^2_{ref}} \frac{\partial p^*}{\partial t^*} 
  - \frac{\lambda_m \: \delta \: T_{ref} \: c_{th}}{\beta_m\:x^2_{ref}} \frac{\partial T^*}{\partial t^*} \\
  + \frac{V_{ref} \: p_{ref}}{x_{ref}}\left[\frac{(1-\phi)\beta_s V^{*(1)}_k + \phi\beta_f V^{*(2)}_k}{\beta_m} \right] \frac{\partial p^*}{\partial x^*_k} \\
  - \frac{V_{ref}\:\delta\:T_{ref}}{x_{ref}}\left[\frac{(1-\phi)\lambda_s V^{*(1)}_k + \phi\lambda_f V^{*(2)}_k}{\beta_m} \right] \frac{\partial T*}{\partial x^*_k} \\
  + \frac{V_{ref}}{\beta_m} \frac{\partial( \phi (V^{*(2)}_k -V^{*(1)}_k))}{\partial x_k}
  + \frac{V_{ref}}{\beta_m} \frac{\partial(V^{*(1)}_k)}{\partial x_k}  
  = \frac{1}{\beta_m} \left(\frac{1}{\rho_f} - \frac{1}{\rho_s}\right)j_1
\end{multline}

This can be rewritten as

\begin{multline}
  \label{eq:mixture_mass_balance5}
  \frac{\partial p^*}{\partial t^*} 
  - \overbrace{\frac{\lambda_m \: \delta \: T_{ref}}{\beta_m\:p_{ref}}}^{\Lambda} \frac{\partial T^*}{\partial t^*} 
  + \overbrace{\frac{x_{ref}\:V_{ref}}{c_{th}}}^{Pe}  \overbrace{\left[\frac{(1-\phi)\beta_s V^{*(1)}_k + \phi\beta_f V^{*(2)}_k}{\beta_m} \right]}^{\vec{v}^p} \frac{\partial p^*}{\partial x^*_k} \\
  - \overbrace{\frac{x_{ref}\:V_{ref}}{c_{th}}}^{Pe}  \frac{\delta\:T_{ref}}{p_{ref}}\overbrace{\left[\frac{(1-\phi)\lambda_s V^{*(1)}_k + \phi\lambda_f V^{*(2)}_k}{\beta_m} \right]}^{\vec{v}^T} \frac{\partial T*}{\partial x^*_k} \\
  + \underbrace{\frac{x_{ref}\:V_{ref}}{c_{th}}}_{Pe} \frac{x_{ref}}{\beta_m\:p_{ref}} \frac{\partial( \phi (V^{*(2)}_k -V^{*(1)}_k))}{\partial x_k}
  + \underbrace{\frac{x_{ref}\:V_{ref}}{c_{th}}}_{Pe} \frac{x_{ref}}{\beta_m\:p_{ref}} \frac{\partial(V^{*(1)}_k)}{\partial x_k}  \\
  = \frac{x^2_{ref}}{\beta_m\:p_{ref}\:c_{th}} \left(\frac{1}{\rho_f} - \frac{1}{\rho_s}\right)j_1
\end{multline}

with

\begin{subequations}
  \label{eq:def_parameters}
  \begin{align}
  \Lambda &= \frac{\lambda_m \: \delta \: T_{ref}}{\beta_m\:p_{ref}}, \\   
  Pe &= \frac{x_{ref}\:V_{ref}}{c_{th}}, \\   
  v^p &= \frac{(1-\phi)\beta_s V^{*(1)}_k + \phi\beta_f V^{*(2)}_k}{\beta_m}, \\   
  v^T &= \frac{(1-\phi)\lambda_s V^{*(1)}_k + \phi\lambda_f V^{*(2)}_k}{\beta_m}.
  \end{align}
\end{subequations}

%\addcontentsline{toc}{chapter}{\numberline{}Bibliography}
%\bibliographystyle{unsrt}
%\begin{thebibliography}{99}
%\bibitem{richards1931}LA Richards ``Capillary conduction of
%  liquids through porous mediums''   Physics 1 (1931) pp 318--333
%\end{thebibliography}





\end{document}

